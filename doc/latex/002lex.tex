\section{Lexical Conventions}
\subsection{Symbols}
At the lowest level, the input is composed of symbols, delimiters and
comments.

{\em Symbols} are made up of any non-empty sequence of
characters from the set of upper and lower case letters,
digits, and the following special characters \verb#.# and \verb#_#.
To specify a symbol not satisfying this restriction, simply put back-quote
(grave accent) characters (`) before and after it.
The string that results after the back-quotes are stripped off then
constitute the symbol.
If the desired symbol contains back-quotes they can be escaped with a
preceding backslash.

The following characters are recognized as delimiters:
\begin{quote}
\begin{verbatim}
! " $ % & ( ) * + , - / : ; < = > ? @ [ \ ] ^ { | } ~
\end{verbatim}
\end{quote}
All of these characters except \verb#" % < > ~# indicate operations or are
used in the specification of operations on automata.

Comments are indicated by a \verb$#$.
All text from this character to the next newline is treated as a comment
and ignored by the parser.

Within expressions, symbols are interpreted as variables, as
primitive tokens (elements of the underlying alphabet), or as filenames.
Two symbol tables are maintained to keep track of variable/token usage.
The variable symbol table contains all symbols that have been assigned
values in assignment statements.
The token symbol table contains all symbols that have been previously used
as tokens.
The printable characters along with the horizontal tab and newline
characters are loaded into this table since they are usually used as tokens.
When a symbol is recognized by the scanner, it is identified as a variable
if it does not occur in the token symbol table.
If the symbol occurs in the variable symbol table, that value is used.
Otherwise the indicated file is read.
The value is the automaton so obtained.

Any symbol beginning with a digit followed by a dot is immediately
identified as a token.
The digit identifies which tape to which the token is assigned.
In the one-tape case, the number is ``0''.
Thus any symbol can be identified as being a token by preceeding it with
the appropriate digit (usually ``0'') and a dot.
The value of the expression indicated by the symbol is an automaton that
recognizes that single token.

If neither of these two cases occurs, the symbol is looked up in the two
symbol tables to determine its past usage.
If it occurs in only one of the symbol tables that determines whether it is
a variable or a token.
If it occurs in neither or both then it is assumed to be a token.
If in both, a warning diagnostic is given.
A warning diagnostic is also provided when an assignment is made to a
symbol in the token table.

In order to display the symbol tables there are user level commands.
The command ``:list'' displays the variable symbol table and ``:alph''
displays the token symbol table.
One will immediately notice that the variable symbol table contains the
symbol \verb#_Last_#.
This refers to the last expression requested that was not assigned to a
variable.
