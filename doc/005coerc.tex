\section{Coercing and Displaying Operators}
\subsection{Coercing Operators}
Internally {\em INR} operates on the following laziness principle.
Each automaton can be in one a number of modes depending on how much work
has been done on it.
At one extreme, the automaton is a collection of transitions appearing any
order, and possibly with duplicate transtions or inaccessible states.
At the other extreme, the automaton is a minimized deterministic automaton
with its transitions sorted and unduplicated.
Each operation then automatically coerces its arguments to the required
form by calling the appropriate routine and sets the mode of its result
value to a appropriate value.
For example, union takes two sets of transitions and after renaming
combines them into one set.
On the other hand, intersection requires that its operands be minimized
deterministic automata and produces as a result a not necessarily minimized
deterministic automaton.
For debugging and education purposes, the coercing operator can be
explicitly used:
\begin{quote}
\begin{description}
\item[:nfa] Sort and unduplicate transitions.
\item[:trim] Remove states that are unreachable. (Reduce)
\item[:lameq] Remove lambda equivalent states.
\item[:lamcm] Combine lambda implied states.
\item[:closed] Form lambda closure.
\item[:dfa] Form deterministic machine. (Subsets Construction)
\item[:min] Minimize the DFA.
\end{description}
\end{quote}
Thus \verb#<expression> :trim;#
will print the trim (reduced) automaton corresponding to the expression.
Note that coercing is one way and that some operators are smart about
promoting their results.
\subsection{Displaying Operators}
The following operators cause some form of printing:
\begin{quote}
\begin{description}
\item[:pr] Print in readable format.
\item[:save] Save in condensed format.
\item[:report] Write a one line report.
\item[:enum] Enumerate words.
\item[:card] Count number of words and print.
\item[:length] Print length of shortest word.
\end{description}
\end{quote}
The operation \verb#:pr# may be followed by a filename in which case the
automaton is printed into that file.
The operation \verb#:save# must be followed by a filename indicating the
desired file.
The operation \verb#:enum# may be followed by a positive number indicating
the limit on printing desired.
The number bounds the length of the longest word (in tokens) and the number
of tokens to be printed.
Thus 1000 means that no word longer than 1000 characters will be printed
and that after 1000 tokens have been displayed no further words will be
started.

\subsection{Input Operator}
To cause an automaton to be loaded the operator \verb#:read# followed by a
filename can be used.
This will load a file in either \verb#:pr# or \verb#:save# format.
