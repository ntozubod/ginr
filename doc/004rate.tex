\section{Generalized Rational Expressions}
A binary relation is a set of ordered pairs of elements.
Thus just as a language can be regarded as a set of words, we can define a
binary language relation as a set of pairs of words.
Binary language relations are often called transductions since they
indicate a transformation or translation process.
Of particular interest are transductions which can be performed by a finite
state machine.
These are called finite state transductions or simply finite transductions.
A two-tape finite automaton or finite transducer determines whether a pair
of words is in the transduction in the following way:
The two words under consideration are placed on two input tapes followed by
special end-marker characters and the machine is put into its start state.
The machine then picks an operation (non-deterministically) which may
involve reading some number of characters from either (or both or neither)
of the two input tapes, advancing the read heads of the two tapes and
making a transition to a new state.
The pair of words is accepted if there exists such a sequence of steps such
that all of both input tapes are read and the transducer is left in a final
state.

As described above the machine is considered to have two input tapes.
However, one of the input tapes can be made into an output tape.
In this case the transducer is viewed as reading an input word and writing
an output word that is non-deterministically chosen from the set of words
related to the input word.
Still another interpretation is to consider the input word as determining a
set of output words that are related to it.

The above description of finite transducers implies a non-deterministic choice
of the next transition to make.
The class of relations that result when the transducer is made
deterministic, either with two input tapes or with an input tape and an
output tape are proper subclasses of finite transductions.
However, it is possible to remove much of the non-determinism using the
idea of a characterizing language.

Before defining a characterizing language we will mark the alphabets with
either a 0 or a 1 to indicate whether they are to be read from tape 0 or
tape 1.
Thus $a$ will occur in the forms $a_0$ and $a_1$.
We can then define languages over this extended alphabet.
From each language we can then obtain a relation by projecting out the
tape~0 letters and the tape~1 letters as the two components of the ordered
pair.
A language then (strongly) characterizes a relation if this projection
process on the language yields exactly the given relation.
Note that each word in the characterizing language is a shuffle of related
marked words.

It is always possible to find a regular characterizing language for any
finite transduction although the choice is not unique and cannot be made
canonical without restricting the class of languages or being
uncomputable.
Thus a convenient way of describing finite transductions is by means of
some regular characterizing language.
Then the usual optimizations and transformations which apply to regular
languages can then be applied without changing the implied transduction.
This is exactly what INR does.
Instead of $a_0$ and $a_1$, INR uses the notation \verb#0.a# and \verb#1.a#.

Regular languages are defined in other ways than by their automata however.
For example regular expressions, especially when extended to include the
Boolean operations are a very expressive way to present regular languages.
In the case of the union operation, this behaviour carries over to
relations since the relation defined as the union of two characterizing
languages is always the union of the relations defined by the two
characterizing languages.
However, there is a slight problem with the other Boolean operations when they
are applied to relations though their characterizing languages.
The result is not usually the same as applying the Boolean operation to the
relation itself.
In fact, the operations intersection, set difference, symmetric difference,
and complement can yield transductions that are not finite state.

Note that we could also define ternary relations as sets of ordered triples
of elements and introduce ternary finite transductions.
We can define characterizing languages by using marking the letters with
numbers 0, 1, or 2.
Again there will always be a regular characterizing language, and similar
closure properties.
The process can, of course, be continued to relations of any degree.
INR currently restricts the degree to 10: \verb#0.a#, \verb#1.a#, $\ldots$,
\verb#9.a#.

Finite transductions are also called rational relations or rational
transductions since they are closed under the rational operations:
concatenation, union, and concatenation closure (Kleene plus).
They are also closed a number of other useful operations which will be
described.
This section outlines the additional operators provided with INR which
allow the definition of rational relations.

\subsection{Token}
To indicate that a token is associated with a particular tape, precede it
by the tape number and a period.
\begin{center}\begin{minipage}[t]{3in}\begin{minipage}[t]{3in}\begin{tabbing}
\qquad \= \verb#1.abc;#\\
or \> \verb#`1.abc`;#
\end{tabbing}\end{minipage}\end{minipage}
\begin{minipage}[t]{1.6in}\begin{verbatim}
(START) 1.abc 2
2 -| (FINAL)
\end{verbatim}\end{minipage}\end{center}
Note that \verb#1.`abc`# will not work because the lexical scanner parses
this as \verb#1.`# followed by \verb#abc# followed by an unmatched
\verb#`#.

\subsection{Variable}
A variable may be assigned a rational relation in the same way that it can
be assigned a regular set.

\subsection{String}
Strings are as for regular languages except for the need to shift them from
tape zero to some other tape.
To define a string on tape 1, there are four basic techniques depending on
the situation.
\begin{enumerate}
\item Build the string from tokens which specify the tape.
Then \verb#'abc'# on tape 1 becomes \verb#1.a 1.b 1.c#.

\item Use (parenthesis/comma) tuple notation.
Then \verb#'abc'# on tape 1 becomes \verb#( ^, 'abc' )# or simply
\verb#(,'abc')#.

\item Use the tape shifting operator \verb#[]#.
Then \verb#'abc'# on tape 1 becomes \verb#['abc']#.

\item Use the tape renumbering operator \verb#$#.
Then \verb#'abc'# on tape 1 is \verb#( 'abc' $ 1 )#.
\end{enumerate}
\begin{center}\begin{minipage}[t]{3in}\begin{minipage}[t]{3in}\begin{tabbing}
\qquad \= \verb#1.a 1.b 1.c;#\\
or \> \verb#(,'abc');#\\
or \> \verb#['abc'];#\\
or \> \verb#('abc' $1);#\\
\end{tabbing}\end{minipage}\end{minipage}
\begin{minipage}[t]{1.6in}\begin{verbatim}
(START) 1.a 2
2 1.b 3
3 1.c 4
4 -| (FINAL)
\end{verbatim}\end{minipage}\end{center}
To indicate a string on tape 2 a similar technique can be used:
\begin{center}\begin{minipage}[t]{3in}\begin{minipage}[t]{3in}\begin{tabbing}
\qquad \= \verb#2.a 2.b 2.c;#\\
or \> \verb#(,,'abc');#\\
or \> \verb#[['abc']];#\\
or \> \verb#('abc' $2);#\\
\end{tabbing}\end{minipage}\end{minipage}
\begin{minipage}[t]{1.6in}\begin{verbatim}
(START) 2.a 2
2 2.b 3
3 2.c 4
4 -| (FINAL)
\end{verbatim}\end{minipage}\end{center}
The empty string is indicated as in the regular case by \verb#^#,
\verb#()#, or \verb#''# and is not associated with any tape.

\subsection{Set Operations}
Rational relations are closed under union and the syntax and behaviour is
as for regular languages.
However, they are not closed under intersection, difference,
symmetric difference or complement so INR interprets these four operations
as applying to the characterizing languages that represent the rational
relations and a warning is printed.
Some advanced uses of INR involving a careful use of this feature will be
discussed in a later section.

\subsection{Concatenation Operations}
The operations concatenation, Kleene plus and star, optional and concatenation
power are as in the regular case.
Concatenation quotient is not defined.
On the other hand, \verb#:alph#, \verb#:rev#, \verb#:pref#, and
\verb#:suff# are available and apply to the characterizing language.

\subsection{Tuple Formation (Cartesian Product)}
The tuple notation consisting of parentheses and commas can be used for
more than just strings.
It computes the number of tapes used by the left operand, adds that number
to each tape number of the right operand and finaly concatenates the two
machines together in the indicated order.
Thus, as far as the characterizing languages are concerned, the comma
operator functions as a concatenation following a tape shift.
\begin{center}\begin{minipage}[t]{3in}\begin{minipage}[t]{3in}\begin{tabbing}
\qquad \= \verb#( a*, b* );#\\
or \> \verb#0.a* 1.b*;#
\end{tabbing}\end{minipage}\end{minipage}
\begin{minipage}[t]{1.6in}\begin{verbatim}
(START) -| (FINAL)
(START) 0.a (START)
(START) 1.b 2
2 -| (FINAL)
2 1.b 2
\end{verbatim}\end{minipage}\end{center}

\subsection{Tape Shifting}
The operator \verb#[]# can be used to cause all of the tapes to be
incremented by one.
\begin{center}\begin{minipage}[t]{3in}\begin{minipage}[t]{3in}\begin{tabbing}
\qquad \= \verb#a* [b*];#\\
or \> \verb#( a*, b* );#\\
or \> \verb#0.a* 1.b*;#
\end{tabbing}\end{minipage}\end{minipage}
\begin{minipage}[t]{1.6in}\begin{verbatim}
(START) -| (FINAL)
(START) 0.a (START)
(START) 1.b 2
2 -| (FINAL)
2 1.b 2
\end{verbatim}\end{minipage}\end{center}
\begin{center}\begin{minipage}[t]{3in}\begin{minipage}[t]{3in}\begin{tabbing}
\qquad \= \verb#[b*] a*;#\\
or \> \verb#(, b* ) a*;#\\
or \> \verb#1.b* 0.a*;#
\end{tabbing}\end{minipage}\end{minipage}
\begin{minipage}[t]{1.6in}\begin{verbatim}
(START) -| (FINAL)
(START) 1.b (START)
(START) 0.a 2
2 -| (FINAL)
2 0.a 2
\end{verbatim}\end{minipage}\end{center}
These two automata describe the same rational relation although, of course,
the characterizing languages are different.

\subsection{Tape Projection}
The \verb#$# operator can be used to reorganize the tape structure of a
transducer.
One such operation is the selection (projection) of some of the tapes with
the implied removal of the others.
Rational relations are closed under all such projections.
For example, for a binary transduction \verb#R1#, it is often useful to
construct an automaton which recognizes the domain of the transduction or
the range.
The domain is projected by INR using the notation \verb#R1 $ 0# and the
range by the notation \verb#R1 $ 1#.
\begin{center}\begin{minipage}[t]{3in}\begin{minipage}[t]{3in}\begin{tabbing}
\qquad \= \verb#(a,b)* $ 0;#\\
or \> \verb#a*;#
\end{tabbing}\end{minipage}\end{minipage}
\begin{minipage}[t]{1.6in}\begin{verbatim}
(START) -| (FINAL)
(START) a (START)
\end{verbatim}\end{minipage}\end{center}
\begin{center}\begin{minipage}[t]{3in}\begin{minipage}[t]{3in}\begin{tabbing}
\qquad \= \verb#(a,b)* $ 1;#\\
or \> \verb#b*;#
\end{tabbing}\end{minipage}\end{minipage}
\begin{minipage}[t]{1.6in}\begin{verbatim}
(START) -| (FINAL)
(START) b (START)
\end{verbatim}\end{minipage}\end{center}
Projections of ternary relations can also be done.
Thus if \verb#R2# is a ternary transduction \verb#R2 $ (0,2)# will select
tapes 0 and 2 renumbering them as 0 and 1.
\begin{center}\begin{minipage}[t]{3in}\begin{minipage}[t]{3in}\begin{tabbing}
\qquad \= \verb#(a,b,c)* $ (0,2);#\\
or \> \verb#(a,c)*;#
\end{tabbing}\end{minipage}\end{minipage}
\begin{minipage}[t]{1.6in}\begin{verbatim}
(START) -| (FINAL)
(START) 0.a 2
2 1.c (START)
\end{verbatim}\end{minipage}\end{center}

\subsection{Tape Renumber}
The \verb#$# operator can also be used to cause a renumbering of the tapes.
For example, the inverse of a binary relation can be formed by applying the
operator \verb#$ (1,0)#.
\begin{center}\begin{minipage}[t]{3in}\begin{minipage}[t]{3in}\begin{tabbing}
\qquad \= \verb#(a,b)* $ (1,0);#\\
or \> \verb#([a] b)*;#
\end{tabbing}\end{minipage}\end{minipage}
\begin{minipage}[t]{1.6in}\begin{verbatim}
(START) -| (FINAL)
(START) 1.a 2
2 0.b (START)
\end{verbatim}\end{minipage}\end{center}
Projection and renumbering can be done together by a natural extension of
the basic idea.
\begin{center}\begin{minipage}[t]{3in}\begin{minipage}[t]{3in}\begin{tabbing}
\qquad \= \verb#(a,b,c)* $ (2,0);#\\
or \> \verb#(a,c)*;#
\end{tabbing}\end{minipage}\end{minipage}
\begin{minipage}[t]{1.6in}\begin{verbatim}
(START) -| (FINAL)
(START) 1.a 2
2 0.c (START)
\end{verbatim}\end{minipage}\end{center}

\subsection{Stretching}
A third use the \verb#$# operator is stretching.
A simple example of this was demonstrated in the last section and the idea
can be generalized to any tape.
\begin{center}\begin{minipage}[t]{3in}\begin{minipage}[t]{3in}\begin{tabbing}
\qquad \= \verb#(a,b)* $ (0 0,1 1 1);#\\
or \> \verb#(a a,b b b)*;#
\end{tabbing}\end{minipage}\end{minipage}
\begin{minipage}[t]{1.6in}\begin{verbatim}
(START) -| (FINAL)
(START) 0.a 2
2 0.a 3
3 1.b 4
4 1.b 5
5 1.b (START)
\end{verbatim}\end{minipage}\end{center}
A more useful type of stretching involves the conversion of an automaton
that accepts a regular language to one that accepts a word in that language
and copies it to the output tape.
\begin{center}\begin{minipage}[t]{3in}\begin{minipage}[t]{3in}\begin{tabbing}
\qquad \= \verb#a b* $ (0,0);#\\
or \> \verb#(a, a) (b, b)*;#
\end{tabbing}\end{minipage}\end{minipage}
\begin{minipage}[t]{1.6in}\begin{verbatim}
(START) 0.a 2
2 1.a 3
3 -| (FINAL)
3 0.b 4
4 1.b 3
\end{verbatim}\end{minipage}\end{center}
Note that the right argument is an automaton that recognizes a string of
tokens of the form {\em digit}.{\em digit} and this may be created in any
way.
For example, if \verb#$ (0,0)# is used often, a variable containing the
value \verb#(0,0)# could be defined to provide an improved readability in
this case.
Note also that how the stretching is done in the multitape case.
The tape numbers are assigned in the order that they occur in the right
argument.
\begin{center}\begin{minipage}[t]{3in}\begin{minipage}[t]{3in}\begin{tabbing}
\qquad \= \verb#a b* $ [0] 0;#\\
or \> \verb#([a] a) ([b] b)*;#
\end{tabbing}\end{minipage}\end{minipage}
\begin{minipage}[t]{1.6in}\begin{verbatim}
(START) 1.a 2
2 0.a 3
3 -| (FINAL)
3 1.b 4
4 0.b 3
\end{verbatim}\end{minipage}\end{center}

\subsection{Composition}
The composition operator is denoted by \verb#@#.
The last tape of the left operand is joined with the first tape of the
right operand.
This yield the usual composition of relations when applied to binary
relations and the usual ``apply'' when the left operand is over one tape
and the right operand is a binary relation.
\begin{center}\begin{minipage}[t]{3in}\begin{minipage}[t]{3in}\begin{tabbing}
\qquad \= \verb#(a,b)* @ (b,c)*;#\\
or \> \verb#(a,c)*;#
\end{tabbing}\end{minipage}\end{minipage}
\begin{minipage}[t]{1.6in}\begin{verbatim}
(START) -| (FINAL)
(START) 0.a 2
2 1.c (START)
\end{verbatim}\end{minipage}\end{center}
If the left argument is a regular language then the compose operator
computes the subset of the range that is the image of this set.
This is a generalization of the application of functions to domain values.
\begin{center}\begin{minipage}[t]{3in}\begin{minipage}[t]{3in}\begin{tabbing}
\qquad \= \verb#'aa' @ (a,b)*;#\\
or \> \verb#'bb';#
\end{tabbing}\end{minipage}\end{minipage}
\begin{minipage}[t]{1.6in}\begin{verbatim}
(START) b 2
2 b 3
3 -| (FINAL)
\end{verbatim}\end{minipage}\end{center}
The compose operator can also be used to compute the subset of the domain
that maps to a subset of the range.
This is really an apply of the inverse relation.
\begin{center}\begin{minipage}[t]{3in}\begin{minipage}[t]{3in}\begin{tabbing}
\qquad \= \verb#(a,b)* @ 'bb';#\\
or \> \verb#'bb' @ ((a,b)* $ (1,0));#\\
or \> \verb#'aa';#
\end{tabbing}\end{minipage}\end{minipage}
\begin{minipage}[t]{1.6in}\begin{verbatim}
(START) a 2
2 a 3
3 -| (FINAL)
\end{verbatim}\end{minipage}\end{center}

\subsection{Join}
The join operator is denoted by \verb#@@#.
It has the same effect as composition except that the last tape of the left
operand is represented in the output.
Thus the composition operator is equivalent to a join followed by a
projection that removes the tape over which the join was done.
The join operator can be used to restrict the domain or range of a
transduction.
\begin{center}\begin{minipage}[t]{3in}\begin{minipage}[t]{3in}\begin{tabbing}
\qquad \= \verb#'aa' @@ (a,b)*;#\\
or \> \verb#(a,b)* @@ 'bb';#\\
or \> \verb#(a,b) :2;#
\end{tabbing}\end{minipage}\end{minipage}
\begin{minipage}[t]{1.6in}\begin{verbatim}
(START) 0.a 2
2 1.b 3
3 0.a 4
4 1.b 5
5 -| (FINAL)
\end{verbatim}\end{minipage}\end{center}

\subsection{Composition Power}
The $k$-fold composition of a binary rational relation is denoted by
\verb#:(#$k$\verb#)#.
That is, the number is written in parentheses.
\begin{center}\begin{minipage}[t]{3in}\begin{minipage}[t]{3in}\begin{tabbing}
\qquad \= \verb#(a,'aa')* :(2);#\\
or \> \verb#(a,'aa')* @ (a,'aa')*;#\\
or \> \verb#(a,'aaaa')*;#
\end{tabbing}\end{minipage}\end{minipage}
\begin{minipage}[t]{1.6in}\begin{verbatim}
(START) -| (FINAL)
(START) 0.a 2
2 1.a 3
3 1.a 4
4 1.a 5
5 1.a (START)
\end{verbatim}\end{minipage}\end{center}

\subsection{Extending Or (ElseOR)}
Often it is desirable to extend one function by a second function, that is,
extend the domain to include that covered by the second function without
disturbing the values assigned by the first function within its domain.
This operator is denoted by the symbol \verb#||#.
Thus if $f$ and $g$ are functions then $f$ \verb#||# $g$ will agree with $f$
whenever the argument is in $dom f$.
Otherwise it will agree with $g$.
Note that \verb#R || S# is simply a shorter form of
\verb#R | ( (S$0)-(R$0) @@ S )#
\begin{center}\begin{minipage}[t]{3in}\begin{minipage}[t]{3in}\begin{tabbing}
\qquad \= \verb#(a,c) || (a,b)*;#\\
or \> \verb#^ | (a,c) | (a,b)(a,b)+;#
\end{tabbing}\end{minipage}\end{minipage}
\begin{minipage}[t]{1.6in}\begin{verbatim}
(START) -| (FINAL)
(START) 0.a 2
2 1.c 3
2 1.b 4
3 -| (FINAL)
4 0.a 5
5 1.b 6
6 -| (FINAL)
6 0.a 5
\end{verbatim}\end{minipage}\end{center}

\subsection{Compute Subsequential}
One interesting subcase of finite transductions is the deterministic
transductions, that is, the transductions that can be performed in one left
to right pass without backtracking or non-deterministic choices with only a
finite number of possible states.
Thus it is desirable to identify when this can be done and construct a
deterministic transducer when it does.
This is performed by the \verb#:sseq# operator.
\begin{center}\begin{minipage}[t]{3in}\begin{minipage}[t]{3in}\begin{tabbing}
\qquad \= \verb#(a,c) || (a,b)* :sseq;#
\end{tabbing}\end{minipage}\end{minipage}
\begin{minipage}[t]{1.6in}\begin{verbatim}
(START) 0.-| 2
(START) 0.a 3
2 1.-| 4
3 0.-| 5
3 0.a 6
4 -| (FINAL)
5 1.c 2
6 1.b 7
7 1.b 8
8 0.-| 2
8 0.a 7
\end{verbatim}\end{minipage}\end{center}
Note that \verb#(START)#, \verb#3# and \verb#8# are read states since they
must read exactly one character (or end-marker) from tape zero.
States \verb#2#, \verb#5#, \verb#6# and \verb#7# are write states.
They can only write and the choice is determined.
Note also that explicit end-markers are introduced.
The existence of explicit end-markers can confuse some of the operations
that use the characterizing language idea.
End-markers can be deleted by using the \verb#$# operator.
It removes explicit end-markers from any tape.
